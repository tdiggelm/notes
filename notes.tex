\documentclass[a4paper, 11pt]{scrartcl}

\usepackage[utf8]{inputenc}
\usepackage[english]{babel}
\usepackage{amsmath}
\usepackage{amsthm}
\usepackage{enumerate}
\usepackage{commath} %pd, od, ...%
\usepackage{mathtools} %\coloneqq, ...%

\newtheorem{lemma}{Lemma}

\title{Physics notes}
\author{Thomas Diggelmann}

\begin{document}
\maketitle

\section*{Maxwell equations}

For the electric field $\textbf{E}$, the magnetic field $\textbf{B}$, the current density $\mathbf{j}$ and the charge density $\rho$, the Maxwell equations read
%
\begin{align*}
  \nabla\cdot\mathbf{E} = \frac{1}{\varepsilon_0}\rho, && \nabla\times\mathbf{E}+\partial_t\mathbf{B}=0, \\
  \nabla\cdot\mathbf{B} = 0,
  && \nabla\times\mathbf{B}+\frac{1}{c^2}\partial_t\mathbf{E}=\mu_0\mathbf{j},
\end{align*}
%
where $c^2 = \frac{1}{\epsilon_0\mu_0}$. The two equations containing the charge and current densities $\rho$ and $\mathbf{j}$ on the RHS are called \emph{inhomogeneous} while the two equations with trivial RHS are called \emph{homogeneous}.

\section*{Continuity and conserved quantities}

A continuity equation is a particular partial differential equation that describes the transport of some quantity $q$. It links the time rate of change of that quantities density $\rho$ with the space rate of change of the quantities current density $\mathbf{j}$
\begin{equation*}
    \nabla\cdot\mathbf{j}+\partial_t\rho=\sigma,
\end{equation*}
where
\begin{itemize}
  \item $\rho$ is the amount of the quantity $q$ per unit volume,
  \item $\mathbf{j}$ is the flux of $q$,
  \item $\sigma$ is the generation of $q$ per unit volume per unit time. Terms that generate ($\sigma > 0$) or remove ($\sigma < 0$) $q$ are referred to as ``sources'' and ``sinks'' respectively.
\end{itemize}
%
If the quantity $q$ is \emph{conserved}, the generation $\sigma$ vanishes and the equation reads
\begin{equation*}
    \nabla\cdot\mathbf{j}+\partial_t\rho=0.
\end{equation*}

\section*{Vector calculus identities}

With $\{\mathbf{x}, \mathbf{A}, \mathbf{B}, \mathbf{C}\} \in \mathbf{R}^n$, $\nabla_i \coloneqq \pd{}{x_i}$ and $r \coloneqq \|\mathbf{x}\| = \sqrt{x_i x_i}$
\begin{enumerate}[(i)]
  \item $\pd{x_i}{x_j}=\delta_{ij}$
  \item $\pd{r}{x_j}=\frac{x_j}{r}$
  \item $\pd{}{x_j}\frac{1}{r}=-\frac{x_j}{r^3}$
  \item $\mathbf{A}\times(\mathbf{B}\times\mathbf{C}) = \mathbf{B}(\mathbf{A}\cdot\mathbf{C})-\mathbf{C}(\mathbf{A}\cdot\mathbf{B})$
  \item $\operatorname{div}(\operatorname{rot}\mathbf{A}) =
  \nabla\cdot(\nabla\times\mathbf{A}) = 0\label{eq:divrot}$
  \item $\operatorname{rot}(\operatorname{grad}\mathbf{\varphi}) =
  \nabla\times(\nabla\mathbf{\varphi}) = \mathbf{0}\label{eq:rotgrad}$
\end{enumerate}

\begin{proof}
  For \eqref{eq:divrot} observe that
  \begin{align*}
    \nabla\cdot(\nabla\times\mathbf{A})
    &= \varepsilon_{ijk} \nabla_i \nabla_j A_k \\
    &= \varepsilon_{jik} \nabla_j \nabla_i A_k
    && \text{Rename $i \leftrightarrow j$} \\
    &= \varepsilon_{jik} \nabla_i \nabla_j A_k
    && \text{Schwarz's theorem} \\
    &= -\varepsilon_{ijk} \nabla_i \nabla_j A_k
    && \text{Antisymmetry of $\varepsilon_{ijk}$} \\
    &= -\nabla\cdot(\nabla\times\mathbf{A}),
  \end{align*}
  and since $\nabla\cdot(\nabla\times\mathbf{A})$ is equal to its negative value it must vanish.
\end{proof}

\section*{Distributions}
Below $\delta$ is the Dirac delta function, $\theta$ is the Heaviside function, $\operatorname{sgn}$ is the Signum function and $\hat{}$ is the Fourier transform operator.
\begin{enumerate}[(i)]
  \item $\theta' = \delta$
  \item $\operatorname{sgn}' = 2\delta\label{eq:dsgn}$
  \item $\hat{1} = (2\pi)^n\delta$
\end{enumerate}

\begin{proof}
  Let $\varphi \in \mathcal{S}(\mathbf{R}^n)$ be a test function.
  For \eqref{eq:dsgn} observe that
  %
  \begin{align*}
    \left<\od{}{x}\operatorname{sgn},\varphi\right> &= -\left<\operatorname{sgn},\od{}{x}\varphi\right> \\
    &=  -\int_\mathbf{R}\operatorname{sgn}(x)\od{}{x}\varphi(x)\operatorname{dx} \\
    &= -(-\varphi(x)\rvert_{-\infty}^0+\varphi(x)\rvert_{0}^\infty) \\
    &= 2\varphi(0) \\
    &= \int_\mathbf{R}\delta(x)\varphi(x)\dif x\\
    &= 2\left<\delta,\varphi\right>.\qedhere
  \end{align*}
\end{proof}

\end{document}
