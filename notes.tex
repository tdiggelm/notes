\documentclass[a4paper, 11pt]{scrartcl}

\usepackage[utf8]{inputenc}
\usepackage[english]{babel}
\usepackage{amsmath}
\usepackage{amsthm}
\usepackage{enumerate}
\usepackage{bm} %bold greek, ...%
\usepackage{commath} %pd, od, ...%
\usepackage{mathtools} %\coloneqq, ...%


\renewcommand{\vec}[1]{\bm{#1}}
\newcommand{\uvec}[1]{\hat{\vec{#1}}}

\newtheorem{lemma}{Lemma}

\title{Physics notes}
\author{Thomas Diggelmann}

\begin{document}
\maketitle

\section*{Maxwell equations}

For the electric field $\vec{E}$, the magnetic field $\vec{B}$, the current density $\vec{j}$ and the charge density $\rho$, the Maxwell equations read
%
\begin{align*}
  \nabla\cdot\vec{E} = \frac{1}{\varepsilon_0}\rho, && \nabla\times\vec{E}+\partial_t\vec{B}=0, \\
  \nabla\cdot\vec{B} = 0,
  && \nabla\times\vec{B}+\frac{1}{c^2}\partial_t\vec{E}=\mu_0\vec{j},
\end{align*}
%
where $c^2 = \frac{1}{\epsilon_0\mu_0}$. The two equations containing the charge and current densities $\rho$ and $\vec{j}$ on the RHS are called \emph{inhomogeneous} while the two equations with trivial RHS are called \emph{homogeneous}.

\section*{Continuity and conserved quantities}

A continuity equation is a particular partial differential equation that describes the transport of some quantity $q$. It links the time rate of change of that quantities density $\rho$ with the space rate of change of the quantities current density $\vec{j}$
\begin{equation*}
    \nabla\cdot\vec{j}+\partial_t\rho=\sigma,
\end{equation*}
where
\begin{itemize}
  \item $\rho$ is the amount of the quantity $q$ per unit volume,
  \item $\vec{j}$ is the flux of $q$,
  \item $\sigma$ is the generation of $q$ per unit volume per unit time. Terms that generate ($\sigma > 0$) or remove ($\sigma < 0$) $q$ are referred to as ``sources'' and ``sinks'' respectively.
\end{itemize}
%
If the quantity $q$ is \emph{conserved}, the generation $\sigma$ vanishes and the equation reads
\begin{equation*}
    \nabla\cdot\vec{j}+\partial_t\rho=0.
\end{equation*}

\section*{Coordinates}

\subsection*{Polar $(r, \varphi)$}

\begin{itemize}
  \item Cartesian coordinates:\\
        $x = r \cos \varphi$\\
        $y = r \sin \varphi$
  \item Unit vectors:\\
        $\uvec{r} = (\cos \varphi, \sin \varphi),$\\
        $\uvec{\varphi} = \uvec{k} \times \uvec{r} = (-\sin \varphi, \cos \varphi)$ \quad (with $\uvec{k}$ normal to plane of motion)
  \item $\vec{r} = (x, y) = r(\cos{\varphi}, \sin \varphi) = r\uvec{r}$,\\
        $\vec{v} = \vec{\dot{r}} = (\dot{x}, \dot{y}) = \dot{r}\uvec{r} + r \dot{\varphi} \uvec{\varphi}$
  \item $\det J = r$ \quad (Jacobian determinant)
  \item $\dif s^2 = \dif r^2+r^2\dif \varphi^2$ \quad (Line element)
  \item $\dif A = \dif x \dif y = r\dif r \dif \varphi$ \quad (Area element)
\end{itemize}

\subsection*{Spherical $(r, \theta, \varphi)$}

\begin{itemize}
  \item $\vec{r} = r \hat{\vec{r}}$\\
        $\vec{v} = \dot{\vec{r}} = \dot{r} \hat{\vec{r}} + r \dot{\theta} \uvec{\theta} + r \dot{\varphi} \sin\theta \uvec{\varphi}$\\
        $\dif \vec{r} = \dif r \uvec{r} + r \dif \theta \uvec{\theta} + r \sin \theta \dif \varphi \uvec{\varphi}$
  \item Cartesian coordinates:\\
        $x = r \sin\theta \cos\varphi$\\
        $y = r \sin\theta \sin\varphi$\\
        $z = r \cos\theta$
  \item $\dif S_r = r^2 sin\theta \dif \theta \dif \varphi$ \quad (Surface element, $\theta$ to $\theta + \dif \theta$, $\varphi$ to $\varphi + \dif \varphi$, fixed radius $r$)
  \item $\dif \Omega = \frac{\dif S_r}{r^2} = \sin\theta \dif \varphi$ \quad (Solid angle)
  \item $\dif V = r^2 \sin\theta \dif r \dif \theta \dif \varphi$ \quad (Volume element)
  \item $\dif s^2 = \dif r^2 + r^2 \dif \theta^2 + r^2 \sin^2 \theta \dif \varphi^2$ (Line element)
\end{itemize}

\section*{Vector calculus identities}

Let $\{\vec{x}, \vec{A}, \vec{B}, \vec{C}\} \in \mathbf{R}^n$, $\nabla_i \coloneqq \pd{}{x_i}$ and $r \coloneqq \|\vec{x}\| = \sqrt{x_i x_i}$ for the following relations.
\begin{enumerate}[(i)]
  \item $\pd{x_i}{x_j}=\delta_{ij}$
  \item $\pd{r}{x_j}=\frac{x_j}{r}$
  \item $\pd{}{x_j}\frac{1}{r}=-\frac{x_j}{r^3}$
  \item $\vec{A}\times(\vec{B}\times\vec{C}) = \vec{B}(\vec{A}\cdot\vec{C})-\vec{C}(\vec{A}\cdot\vec{B})$
  \item $\operatorname{div}(\operatorname{rot}\vec{A}) =
  \nabla\cdot(\nabla\times\vec{A}) = 0\label{eq:divrot}$
  \item $\operatorname{rot}(\operatorname{grad}\vec{\varphi}) =
  \nabla\times(\nabla\vec{\varphi}) = \vec{0}\label{eq:rotgrad}$
\end{enumerate}

\begin{proof}
  For \eqref{eq:divrot} observe that
  \begin{align*}
    \nabla\cdot(\nabla\times\vec{A})
    &= \varepsilon_{ijk} \nabla_i \nabla_j A_k \\
    &= \varepsilon_{jik} \nabla_j \nabla_i A_k
    && \text{Rename $i \leftrightarrow j$} \\
    &= \varepsilon_{jik} \nabla_i \nabla_j A_k
    && \text{Schwarz's theorem} \\
    &= -\varepsilon_{ijk} \nabla_i \nabla_j A_k
    && \text{Antisymmetry of $\varepsilon_{ijk}$} \\
    &= -\nabla\cdot(\nabla\times\vec{A}),
  \end{align*}
  and since $\nabla\cdot(\nabla\times\vec{A})$ is equal to its negative value it must vanish.
\end{proof}

\section*{Distributions}
Below $\delta$ is the Dirac delta function, $\theta$ is the Heaviside function, $\operatorname{sgn}$ is the Signum function and $\hat{}$ is the Fourier transform operator.
\begin{enumerate}[(i)]
  \item $\theta' = \delta$
  \item $\operatorname{sgn}' = 2\delta\label{eq:dsgn}$
  \item $\hat{1} = (2\pi)^n\delta$
\end{enumerate}

\begin{proof}
  Let $\varphi \in \mathcal{S}(\mathbf{R}^n)$ be a test function.
  For \eqref{eq:dsgn} observe that
  %
  \begin{align*}
    \left<\od{}{x}\operatorname{sgn},\varphi\right> &= -\left<\operatorname{sgn},\od{}{x}\varphi\right> \\
    &=  -\int_{\mathbf{R}}\operatorname{sgn}(x)\od{}{x}\varphi(x)\operatorname{dx} \\
    &= -(-\varphi(x)\rvert_{-\infty}^0+\varphi(x)\rvert_{0}^\infty) \\
    &= 2\varphi(0) \\
    &= \int_{\mathbf{R}}\delta(x)\varphi(x)\dif x\\
    &= 2\left<\delta,\varphi\right>.\qedhere
  \end{align*}
\end{proof}

\end{document}
